 \documentclass[a4paper,11pt]{article}
\usepackage{epstopdf, epsfig}
\usepackage{amsmath}
\usepackage{amssymb}
%\usepackage{amsthm}
\usepackage{graphicx}
%\usepackage{caption}
%\usepackage{subcaption}

\newcommand{\ba}{\begin{array}}
\newcommand{\ea}{\end{array}}

\newcommand{\bea}{\begin{eqnarray}}
\newcommand{\eea}{\end{eqnarray}}

\newcommand{\bc}{\begin{center}}
\newcommand{\ec}{\end{center}}

\newcommand{\ds}{\displaystyle}

\newcommand{\bt}{\begin{tabular}}
\newcommand{\et}{\end{tabular}}

\newcommand{\bi}{\begin{itemize}}
\newcommand{\ei}{\end{itemize}}

\newcommand{\bd}{\begin{description}}
\newcommand{\ed}{\end{description}}

\newcommand{\bp}{\begin{pmatrix}}
\newcommand{\ep}{\end{pmatrix}}

\newcommand{\pd}{\partial}
\newcommand{\sech}{\mbox{sech}}

\newcommand{\cf}{{\it cf.}~}

\newcommand{\ltwo}{L_{2}(\mathbb{R}^{2})}
\newcommand{\smooth}{C^{\infty}_{0}(\mathbb{R}^{2})}

\newcommand{\br}{{\bf r}}
\newcommand{\bk}{{\bf k}}
\newcommand{\bv}{{\bf v}}

\newcommand{\gnorm}[1]{\left|\left| #1\right|\right|}
\newcommand{\ipro}[2]{\left<#1,#2 \right>}

%\setlength{\topmargin}{-40pt} \setlength{\oddsidemargin}{0pt}
%\setlength{\evensidemargin}{0pt} \setlength{\textwidth}{460pt}
%\setlength{\textheight}{680pt}

\begin{document}
So we are beating
\[
i\psi_{t} = -\Delta \psi + \left| \psi\right|^{2}\psi + \gamma(x,y,t)
\]
to death.  If we are looking for periodic coefficients on a periodic box $[-L,L]\times[-L,L]$, then we are supposing that
\[
\psi(x,y,t) = \sum_{n,m}\hat{\psi}_{nm}(t) e^{i\frac{\pi}{L}(mx + ny)}, 
\]
where
\[
 \hat{\psi}_{nm}(t) = \frac{1}{(2L)^{2}}\int_{-L}^{L}\int_{-L}^{L} \psi(x,y,t) e^{-i\frac{\pi}{L}(mx + ny)} dxdy.
\]
At this point, if we approximate by taking a finite number of modes in $n$ and $m$, say $-K+1\leq n \leq K$, so that the total number of modes along each direction is given by $K_{T}=2K$, we can talk about two different meshes.  In physical space, we have a mesh of width $\delta x = L/K$, while in frequency space, we have a mesh of width $\delta k = \pi/L$.  Once we discretize in physical space, we then find, using the Trapezoid method, the approximation to the Fourier coefficients $\hat{\psi}_{nm}$ so that 
\[
\hat{\psi}_{nm}(t) \sim \frac{1}{K_{T}^{2}}\sum_{j,k}\psi_{jk}e^{-i\frac{\pi}{L}(mx_{j} + ny_{k})}
\]
We note though, that in Matlab, the FFT implementation places the $1/K_{T}^{2}$ on the inverse transform.  So, to compare Nazarenko and Onorato, we need to multiply our Fourier coefficients by $K_{T}^{2}$.  

Next, if are going to work in magnitudes of $k$-space vectors, then if we define ${\bf k}_{nm} = \delta k \left(n,m \right)$, then 
\[
k_{nm} = \delta k \sqrt{n^{2}+m^{2}}
\]
Note, when we find 
\[
n({\bf k}_{nm}) = \overline{\left| \hat{\psi}_{nm}\right|^{2}}, 
\]
Nazarenko and Onorato assume that $n$ is isotropic, and thus independent of angle in $k$ space.  
\end{document}