\documentclass[a4paper,11pt]{article}

\usepackage{amsmath}
\usepackage{amssymb}
%\usepackage{amsthm}
\usepackage{graphicx}
\usepackage{epstopdf}
\epstopdfsetup{update}
%\usepackage{caption}
%\usepackage{subcaption}

\newcommand{\ba}{\begin{array}}
\newcommand{\ea}{\end{array}}

\newcommand{\bea}{\begin{eqnarray}}
\newcommand{\eea}{\end{eqnarray}}

\newcommand{\bc}{\begin{center}}
\newcommand{\ec}{\end{center}}

\newcommand{\ds}{\displaystyle}

\newcommand{\bt}{\begin{tabular}}
\newcommand{\et}{\end{tabular}}

\newcommand{\bi}{\begin{itemize}}
\newcommand{\ei}{\end{itemize}}

\newcommand{\bd}{\begin{description}}
\newcommand{\ed}{\end{description}}

\newcommand{\bp}{\begin{pmatrix}}
\newcommand{\ep}{\end{pmatrix}}

\newcommand{\pd}{\partial}
\newcommand{\sech}{\mbox{sech}}

\newcommand{\cf}{{\it cf.}~}

\newcommand{\ltwo}{L_{2}(\mathbb{R}^{2})}
\newcommand{\smooth}{C^{\infty}_{0}(\mathbb{R}^{2})}

\newcommand{\br}{{\bf r}}
\newcommand{\bk}{{\bf k}}
\newcommand{\bv}{{\bf v}}

\newcommand{\gnorm}[1]{\left|\left| #1\right|\right|}
\newcommand{\ipro}[2]{\left<#1,#2 \right>}

\author{Christopher W. Curtis \\
Ricardo Carretero\\
Matteo Polimeno}
\date{}
\title{Characterizing Coherent Structures in Bose-Einstein Condensates through Dynamic-Mode Decomposition}
\begin{document}
\maketitle
\section*{Introduction}
With the recent experimental observation of turbulent cascades in a Bose-Einstein Condensate (BEC) \cite{navon}, it is important to continue to better understand and characterize in as quantitative a means as possible the complex dynamics associated with turbulence in dispersive, nonlinear-wave systems.  While small-amplitude states whose statistics remain nearly Gaussian permit a relatively complete analytic characterization of turbulent cascades, embodied in the weak-wave turbulence (WWT) theory initiated in \cite{zakharov} and collected in \cite{nazarenko}, as noted in \cite{newell,cai}, the assumptions which one makes to derive results in WWT necessarily must generically break down over long-enough timescales.  

This breakdown is best characterized by the formation of long-wavelength, larger-amplitude coherent structures (CSs).  In classic, one-dimensional systems, characterizing such structures in terms of solitons is relatively straightforward; see \cite{cai}.  However, in two-dimensions, and for systems like the defocusing nonlinear Schr\"{o}dinger equation (NLSE), describing coherent structures in quantitative terms is far more challenging.  In the context of BECs, a variety of heuristic metrics to characterize CSs appeared in \cite{nazarenko2}.  In the broader context of WWT, along with classic approaches built around studying qualitative features in Fourier transforms, methods based on tracking spikes in Gaussian curvature of the solution have appeared in \cite{mordant}.  

However, as noted in \cite{nazarenko2}, CSs are difficult to visulalize in terms of physically measurable variables in BECs.  This is due in part to the role that vortices play in BECs, whereby the formation of CSs corresponds to the elimination of vortices.  This in some sense removes the most readily identifiable features of the flow, making characterization of the transition away from the WWT state difficult.  In this paper, instead, we study the use of Dynamic-Mode Decompositions (DMDs), \cite{schmid,williams,kutz}, which is a modal decomposition generated by discrete snap-shots of the temporal evolution of the BEC.  The advantage of DMDs is in their great flexibility due to their essentially being a model-free means of analyzing flows.  

As we show, by selecting the most temporally dominant modes from the DMD, we are readily able to characterize coherent states that otherwise remain undetectable in the BEC flow.  

\section*{Modelling and Weak-Wave Turbulence}

In non-dimensional coordinates (see the Appendix for details on the non-dimensionalization), we model the BEC through the use of a stochastically forced Gross--Pitaevskii (GP), or NLS, equation 
\[
i\psi_{t} = -\Delta \psi +  \left| \psi\right|^{2}\psi + \gamma_{f}({\bf x},t) - i(\nu_{h}\Delta^{2n}+\nu_{l}\tilde{\Delta}^{-2n})\psi, ~ \psi({\bf x},0)=0.
\]
Note, we always remain in the defocusing, or `dark', case.  We solve this equation with periodic boundary conditions, where the common period $2L \gg 1$ so that we have the equivalent Fourier representation of $\psi$,
\[
\psi({\bf x},t) = \sum_{{\bf k}_{mn}} a({\bf k}_{mn},t) e^{\pi i {\bf k}_{mn}\cdot {\bf x}/L}.
\]

The forcing $\gamma_{f}$ is chosen so as to be a spectrally-band limited function 
\[
\gamma_{f}({\bf x},t) = \gamma_{0}e^{2\pi i\varphi(t)}\sum_{k_{l}\leq |{\bf k}_{mn}| \leq k_{h}} e^{\pi i {\bf k}_{mn}\cdot {\bf x}/L}, ~ {\bf k}_{mn} = (m,n), 
\]
and the phase $\varphi(t)$ is such that $\varphi(t)  \sim U(0,1)$ where $U(0,1)$ denotes random variables uniformly distributed between $0$ and $1$.  Thus, our forcing is characterized by an injection range of wavenumbers via the choices of $k_{l}$ and $k_{h}$.  We likewise see that the forcing is unbiased in any particular spatial direction, so that by starting with zero-initial conditions, we see that the solution $\psi({\bf x},t)$ will largely mimick the forcing until it has reached a large enough amplitude that nonlinearity, through four-wave mixing, is able to transfer energy across Fourier modes.  This is ultimately balanced against the strength of the hyperviscosity characterized by the magnitude of $\nu_{h}$ and the hypoviscosity characterized by the magnitude of $\nu_{l}$.  

We note that the question of what a `large' domain is is somewhat ambiguous in this problem due to the forcing.  Traditionally when modelling a BEC, a length scale is set via the `healing-length' \cite{pethick}, which ultimately determines the width of vortices in the GPE.  However, to do this one must have a fixed particle number $N= \int \left|\psi\right|^{2}d{\bf x}$, but due to the forcing we necessarily have that 
\begin{align*}
\frac{1}{2}\frac{dN}{dt} = &  \mbox{Im}\left\{\int \gamma_{f}({\bf x},t) \psi^{\ast} d{\bf x} \right\}\\
& - \int \psi^{\ast}\left(\nu_{h}\Delta^{2n}+\nu_{l}\tilde{\Delta}^{-2n}\right)\psi d{\bf x}
\end{align*}
so that the particle number, and thus length scale, can change with time.  Ultimately though, a quasi-equilibrium is achieved through the balance of injection due to the forcing $\gamma_{f}$ and the particle removal/energy dissipation due to the hyper and hypoviscosity.  

\section*{Dynamic-Mode Decomposition}

\section*{Characterizing Coherent States}

\subsection*{Down-Scale Weak Turbulence Cascades}

\subsection*{Long-Wavelength Saturation and Coherent States}

\section*{Conclusion}

\section*{Appendix}
With units, our model of a BEC is given by the following Gross--Pitaevksii equation (GPE)
\[
i\hbar\psi_{t} = -\frac{\hbar^{2}}{2m}\Delta \psi + g\left| \psi\right|^{2}\psi, ~ g = \frac{4\pi \hbar^{2}a_{s}}{m}
\]
where $a_{s}$ is the `scattering length', and with the clear understanding that $\left|\psi\right|^{2}dxdy$ describes probabilities in the sense that 
\[
\int |\psi|^{2}dxdy = N,
\]
where we take $N$ to be the total number of particles under consideration.  Introducing the non-dimensionalizations 
\[
\tilde{x} = x/\lambda, ~ \tilde{y} = y/\lambda, ~ \tilde{t} = t/T, 
\]
and choosing
\[
\lambda^{2} = \frac{1}{8\pi |a_{s}|}, ~ T = \frac{m}{4\pi\hbar |a_{s}|}, 
\] 
then gives us
\[
i\psi_{t} = -\Delta \psi + \sigma\left| \psi\right|^{2}\psi,  ~\sigma = \mbox{sgn}(a_{s}).
\]

\bibliography{wwt}
\bibliographystyle{unsrt}

\end{document}